\documentclass[12pt]{article}
\usepackage{latexsym,amssymb,amsmath} % for \Box, \mathbb, split, etc.
% \usepackage[]{showkeys} % shows label names
\usepackage{cite} % sorts citation numbers appropriately
\usepackage{path}
\usepackage{url}
\usepackage{verbatim}
\usepackage[pdftex]{graphicx}
\usepackage[colorlinks=true,linkcolor=blue,citecolor=red,urlcolor=blue,pdfauthor={Paul Tune},pdftitle={Fisher Information Analysis of Error 
Estimation Codes}]{hyperref}

% horizontal margins: 1.0 + 6.5 + 1.0 = 8.5
\setlength{\oddsidemargin}{0.0in}
\setlength{\textwidth}{6.5in}
% vertical margins: 1.0 + 9.0 + 1.0 = 11.0
\setlength{\topmargin}{0.0in}
\setlength{\headheight}{12pt}
\setlength{\headsep}{13pt}
\setlength{\textheight}{625pt}
\setlength{\footskip}{24pt}

\renewcommand{\textfraction}{0.10}
\renewcommand{\topfraction}{0.85}
\renewcommand{\bottomfraction}{0.85}
\renewcommand{\floatpagefraction}{0.90}

\makeatletter
\setlength{\arraycolsep}{2\p@} % make spaces around "=" in eqnarray smaller
\makeatother

% change equation, table, figure numbers to be counted inside a section:
\numberwithin{equation}{section}
\numberwithin{table}{section}
\numberwithin{figure}{section}

%------- My common Definitions
\newcommand{\be}{\begin{equation}}
\newcommand{\ee}{\end{equation}}
\newcommand{\ben}{\begin{equation*}}
\newcommand{\een}{\end{equation*}}
\newcommand{\ba}{\begin{eqnarray}}
\newcommand{\ea}{\end{eqnarray}}
\newcommand\Var {{\rm Var}}
\def\Cov {\makebox{Cov }}
\newcommand{\Z}{Z\!\!\!Z}

%%%%%%%%%%%%%%%%% definitions for this document
\def\mus{{$\mu$s}}         % microseconds
% THEOREMS -------------------------------------------------------
\newtheorem{thm}{Theorem}%[section]
\newtheorem{cor}[thm]{Corollary}
\newtheorem{lem}[thm]{Lemma}
\newtheorem{prop}[thm]{Proposition}
\newtheorem{defn}[thm]{Definition}
\newtheorem{rem}[thm]{Remark}
% MATH -----------------------------------------------------------
\newcommand{\id}[1]{\mathbf{I}_{#1}}
\newcommand{\mnull}{\mathbf{0}}
\newcommand{\vnull}[2]{\mathbf{0}_{#1 \times #2}}
\newcommand{\norm}[1]{\left\Vert#1\right\Vert}
\newcommand{\abs}[1]{\left\vert#1\right\vert}
\newcommand{\set}[1]{\left\{#1\right\}}
\newcommand{\mset}[1]{\lbrack #1\rbrack}
\newcommand{\Real}{\mathbb R}
\newcommand{\Complex}{\mathbb C}
\newcommand{\Integer}{\mathbb Z}
\newcommand{\Natural}{\mathbb N}
\newcommand{\Sphere}{\mathbb S}
\newcommand{\eps}{\varepsilon}
\newcommand{\To}{\longrightarrow}
\newcommand{\BX}{\mathbf{B}(X)}
\newcommand{\I}{\boldsymbol{\mathcal{I}}_{\vth}}
\newcommand{\tr}{\mathbf{tr}}
\newcommand{\adj}{\mathrm{adj}}
\newcommand{\T}{\mathrm{T}}
\newcommand{\E}[2][]{\mathbb E_{#1} \!\left[ #2 \right]} % optional extra parameter #1
\newcommand{\bEu}[1]{\mathbf{e}_{#1}}
\newcommand{\bin}[2]{\text{Bin}(#1,#2)}
\newcommand{\Count}{1}
\newcommand{\constr}{\mathbf{G}}
\newcommand{\ind}{\mathbbm 1}
\newcommand{\ones}[1]{\mathbf{1}_{#1}}
\newcommand{\veth}{\boldsymbol{\hat \theta}}
\newcommand{\vbth}{\boldsymbol{\bar \theta}}
\newcommand{\iset}[1]{\lbrack #1 \rbrack}
\newcommand{\bxi}[1]{\mathbf{x}^{(#1)}}
\newcommand{\bXi}[1]{\mathbf{X}^{(#1)}}
\newcommand{\iomega}[1]{\omega^{(#1)}}
\newcommand{\bdi}[1]{\mathbf{d}^{(#1)}}
\newcommand{\argmax}{\operatornamewithlimits{argmax}}
% ----------------------------------------------------------------
\renewcommand{\labelenumi}{(\roman{enumi})}
\newcounter{tempcnt}
\renewcommand{\arraystretch}{1.2}
%-----------------------------------------------------------------

\def\gap{\vspace{10pt}\noindent}
\def\naive{na\"{\i}ve\ }

\def\diag{\text{diag}}

\def\bA{\mathbf{A}}
\def\bC{\mathbf{C}}
\def\bD{\mathbf{D}}
\def\bE{\mathbf{E}}
\def\bH{\mathbf{H}}
\def\bM{\mathbf{M}}
\def\bN{\mathbf{N}}
\def\bP{\mathbf{P}}
\def\bW{\mathbf{W}}
\def\bU{\mathbf{U}}
\def\bQ{\mathbf{Q}}
\def\bV{\mathbf{V}}
\def\bF{\mathbf{F}}
\def\bS{\mathbf{S}}
\def\bT{\mathbf{T}}
\def\bX{\mathbf{X}}
\def\bY{\mathbf{Y}}
\def\bZ{\mathbf{Z}}
\def\bB{\mathbf{B}}

\def\bp{\mathbf{p}}

\def\bgamma{\boldsymbol{\gamma}}
% set two lengths for the includegraphics commands used to import the plots:
\newlength{\fwtwo} \setlength{\fwtwo}{0.45\textwidth}
% end of personal macros

\begin{document}
\DeclareGraphicsExtensions{.jpg}

\begin{center}
\textbf{\Large Fisher Information Analysis of Error Estimation Codes} \\[6pt]
  Paul Tune \\[6pt]
  School of Mathematical Sciences,\\
  University of Adelaide, Australia  \\[6pt]
  paul.tune@adelaide.edu.au
\end{center}

\begin{abstract}
In this technical report, we analyze the Fisher information of the generalized Error Estimating Codes (EEC) when the packet is sampled with 
and without replacement, in the case when the algorithm has immunity. We also compute the best bound possible obtained from looking 
directly at the Fisher information of transmitted bits.
\end{abstract}

\subparagraph{Keywords.} Error estimation codes, Fisher information, sampling with and without replacement,
sketching.


%\section{Introduction}

\section{Fisher Information from Direct Samples}

Our model is a two level model: we start from the bits \textit{sampled} from a packet, transmitted across a channel with \textit{bit flip probability} 
$0 < \theta < 1$ and received at the decoder. In this section we consider two important issues:
\begin{enumerate}
\item what is the Fisher information about $\theta$ from directly received samples?
\item what is the difference in Fisher information when bits from the packet are sampled \textit{with} versus \textit{without} replacement?
\end{enumerate}

To answer both questions, we have to develop a framework. Assume $\ell$ bits were sampled from the packet. At this point, we do not specify
the sampling method. Let $\Omega$ be the ordered set of the $2^\ell$ possible bit sequences obtained from sampling from the packet. The
lexicographical ordering starts from $\{0,1,\cdots,2^{\ell}-1\}$, with its binary representation corresponding to $\{00\ldots0, 00\ldots1,\cdots,
11\ldots1\}$. The associated probability mass function (p.m.f.) of the members of the ordered set is denoted by $p_i$, $i = 0,1,\cdots,
2^{\ell}-1$. For example, the probability the all-zero sequence $00\ldots0$ was drawn from the packet is given by $p_0$, while the all ones 
sequence $11\ldots1$ was drawn is $p_{2^\ell-1}$.

Let $X \in \Omega$ be the sampled sequence of bits. These bits are transmitted across the channel and are subjected to bit flip errors due to the
action of the channel. Let the random variable $Y$ denote the received sequence of bits at the decoder. Clearly, $Y \in \Omega$. For each bit 
sequence, we can define a $2^\ell$ by $2^\ell$ matrix $\bC(\theta)$ with entries $c_{i,j}(\theta)$ such that
\begin{align}
\nonumber
c_{i,j}(\theta) &:= \Pr(Y= j\,|\,X=i)\\
\label{eq:channel_error}
&\ = \theta^{h(i,j)} (1-\theta)^{\ell - h(i,j)},
\end{align}
where $h(i,j)$ is the Hamming weight between $i$ and $j$ (in their respective binary representations). The differential of each entry with
respect to $\theta$ is
\begin{align}
\nonumber
c'_{i,j}(\theta) &= \frac{d}{d\theta} c_{i,j}(\theta)\\
\label{eq:channel_error_diff}
&= h(i,j) \theta^{h(i,j)-1} (1-\theta)^{\ell - h(i,j)} - (\ell - h(i,j)) \theta^{h(i,j)} (1-\theta)^{\ell - h(i,j)-1},
\end{align}

Let $\bp = [p_0,p_1,\cdots,p_{2^\ell-1}]$.
The joint probability of $(X,Y)$ is the matrix $\bB(\theta) = \diag(\bp) \bC(\theta)$.
Then, letting $f_{i,j}(\theta) := \Pr(X=i,Y=j;\theta)$ and $f_{j}(\theta) := \Pr(Y=j;\theta)$,
\begin{align*}
f_j(\theta) &= \sum_{i=0}^{2^\ell-1} b_{i,j}(\theta)\\
&= \sum_{i=0}^{2^\ell-1} p_i c_{i,j}(\theta).
\end{align*}

%Let $[\bA]_{\ast,j}$ be column $j$ of the matrix $\bA$ and $\ones$ be a column vector of ones. 


Let $J(X,Y;\theta)$ be the Fisher information of the joint random variables $(X,Y)$. This is the best we can do since in this case, we know what
was sent through the channel in addition to what was received at the decoder. As a side note, when deriving the Fisher information, all 
derivations have to account for the inequality constraint on the parameter $\theta$. However, since $\theta$ does not lie on the boundary of the 
interval $[0,1]$, this does not add additional information, so the Fisher information remains the same as the case when there is no constraint on 
$\theta$ \cite{Gorman90CRB}.

The Fisher information in this case is
\begin{align}
\nonumber
J(X,Y;\theta) &= \sum_{i,j} f_{i,j}(\theta) \Big(\frac{d}{d\theta}\log f_{i,j}(\theta)\Big)^2\\
\nonumber
&= \sum_{i,j} p_i c_{i,j}(\theta) \frac{p^2_i c'^2_{i,j}(\theta)}{p^2_i c^2_{i,j}(\theta)}\\
\label{eq:FI_joint}
& = \sum_{i,j}  p_i\frac{c'^2_{i,j}(\theta)}{c_{i,j}(\theta)}.
\end{align}
On the other hand,
\begin{align}
\nonumber
J(Y;\theta) &= \sum_{j}  f_{j}(\theta) \Big(\frac{d}{d\theta}\log f_{j}(\theta)\Big)^2\\
\label{eq:FI_received}
&= \sum_{j} \Big(\sum_i p_i c'_{i,j}(\theta) \Big)^2 \Big(\sum_i p_i c_{i,j}(\theta) \Big)^{-1} .
\end{align}
By the data processing inequality for Fisher information \cite{Zamir98DPI}, since $\theta \to (X,Y) \to Y$ forms a Markov chain, so
$J(X,Y;\theta) \ge J(Y;\theta)$. 

Now, to answer the second question, we analyze $J(X,Y;\theta)$ further. Let $p^{(wr)}_i$ be the probability sequence $i$ was drawn
with replacement from the packet, while $p^{(wor)}_i$ represents the probability it was drawn from the packet without replacement. Let 
$J(X^{(wr)},Y;\theta)$ and $J(X^{(wor)},Y;\theta)$ be the Fisher information when the bits were drawn with and without replacement 
respectively.


\section{Generalized EEC (gEEC)}

\subsection{Sampling with vs. without replacement}

Intuitively, it may seem that if one samples bits from the packet \textit{without replacement}, there is additional information in the subsketches 
due to the non-avoidance property of the sampling method. However, our aim in this section is to show that there is absolutely no difference
in the information content if \textit{sampling with replacement} was used instead.


\subsection{Computation of Fisher information}

Here we show a fast way of computing the Fisher information for gEEC, based on \cite[Lemma 1]{Hua12gEEC}. Thus, diagonalization of a 
matrix is not required. The trick is to realize that $\bM(\theta)$ in \cite[Equation (26), Lemma 1]{Hua12gEEC} is a circular matrix, so multiplication 
is equivalent to circular convolution.

Let $\ast_K$ denote the circular convolution operation truncated at $K$ and $\bgamma(\theta,0) = [1,0,\cdots,0]$. Then, 
\be
\bgamma(\theta,\ell) = \bgamma(\theta,\ell-1) \ast_K M(\theta),
\ee
where $M(\theta) := [1-\theta,\theta/2,\cdots 0 \cdots,\theta/2]$.  Similarly, by the chain rule, since 
the circular convolution is a linear operation (multiplication by a circulant matrix),
\begin{align*}
\bgamma'(\theta,\ell) &:= \frac{d}{d\theta} \bgamma(\theta,\ell)\\
&= \bgamma'(\theta,\ell-1) \ast_K m(\theta) + \bgamma(\theta,\ell-1) \ast_K M'(\theta),
\end{align*}
where $M'(\theta) :=  [-1,1/2,\cdots 0 \cdots,1/2]$. Clearly, $\bgamma'(\theta,0) = [0,0,\cdots,0]$.

The Fisher information is simply
\be
J(\theta,\ell) = \sum_{k=0}^{K-1} J_k(\theta,\ell)
\label{eq:FI_gEEC}
\ee
with
\be
J(\theta,\ell) := 
\begin{cases}
\frac{(\gamma'_k(\theta,\ell))^2}{\gamma_k(\theta,\ell)}, & \gamma_k(\theta,\ell) > 0,\\
0, & \text{otherwise}.
\end{cases}
\ee
We can compute these relations via publicly available circular convolution routines. Computational complexity of both $\bgamma(\theta,\ell)$ 
and $\bgamma'(\theta,\ell)$ are $O(K \log K)$ respectively, and since by \eqref{eq:FI_gEEC}, the Fisher information computation involves
the summation of $K$ terms, the computation is dominated by the convolution operation, hence the total complexity is $O(K \log K)$.

 
\appendix

\bibliography{TuneBib}
\bibliographystyle{abbrv}

\end{document}

