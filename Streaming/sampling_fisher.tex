\documentclass[12pt]{article}
\usepackage{latexsym,amssymb,amsmath} % for \Box, \mathbb, split, etc.
% \usepackage[]{showkeys} % shows label names
\usepackage{cite} % sorts citation numbers appropriately
\usepackage{path}
\usepackage{url}
\usepackage{verbatim}
\usepackage[pdftex]{graphicx}
\usepackage[colorlinks=true,linkcolor=blue,citecolor=red,urlcolor=blue,pdfauthor={Paul Tune},pdftitle={Fisher Information for Error Estimation Codes}]{hyperref}

% horizontal margins: 1.0 + 6.5 + 1.0 = 8.5
\setlength{\oddsidemargin}{0.0in}
\setlength{\textwidth}{6.5in}
% vertical margins: 1.0 + 9.0 + 1.0 = 11.0
\setlength{\topmargin}{0.0in}
\setlength{\headheight}{12pt}
\setlength{\headsep}{13pt}
\setlength{\textheight}{625pt}
\setlength{\footskip}{24pt}

\renewcommand{\textfraction}{0.10}
\renewcommand{\topfraction}{0.85}
\renewcommand{\bottomfraction}{0.85}
\renewcommand{\floatpagefraction}{0.90}

\makeatletter
\setlength{\arraycolsep}{2\p@} % make spaces around "=" in eqnarray smaller
\makeatother

% change equation, table, figure numbers to be counted inside a section:
\numberwithin{equation}{section}
\numberwithin{table}{section}
\numberwithin{figure}{section}

%------- My common Definitions
\newcommand{\be}{\begin{equation}}
\newcommand{\ee}{\end{equation}}
\newcommand{\ben}{\begin{equation*}}
\newcommand{\een}{\end{equation*}}
\newcommand{\ba}{\begin{eqnarray}}
\newcommand{\ea}{\end{eqnarray}}
\newcommand\Var {{\rm Var}}
\def\Cov {\makebox{Cov }}
\newcommand{\Z}{Z\!\!\!Z}

%%%%%%%%%%%%%%%%% definitions for this document
\def\mus{{$\mu$s}}         % microseconds
% THEOREMS -------------------------------------------------------
\newtheorem{thm}{Theorem}%[section]
\newtheorem{cor}[thm]{Corollary}
\newtheorem{lem}[thm]{Lemma}
\newtheorem{prop}[thm]{Proposition}
\newtheorem{defn}[thm]{Definition}
\newtheorem{rem}[thm]{Remark}
% MATH -----------------------------------------------------------
\newcommand{\id}[1]{\mathbf{I}_{#1}}
\newcommand{\mnull}{\mathbf{0}}
\newcommand{\vnull}[2]{\mathbf{0}_{#1 \times #2}}
\newcommand{\norm}[1]{\left\Vert#1\right\Vert}
\newcommand{\abs}[1]{\left\vert#1\right\vert}
\newcommand{\set}[1]{\left\{#1\right\}}
\newcommand{\mset}[1]{\lbrack #1\rbrack}
\newcommand{\Real}{\mathbb R}
\newcommand{\Complex}{\mathbb C}
\newcommand{\Integer}{\mathbb Z}
\newcommand{\Natural}{\mathbb N}
\newcommand{\Sphere}{\mathbb S}
\newcommand{\eps}{\varepsilon}
\newcommand{\To}{\longrightarrow}
\newcommand{\BX}{\mathbf{B}(X)}
\newcommand{\I}{\boldsymbol{\mathcal{I}}_{\vth}}
\newcommand{\tr}{\mathbf{tr}}
\newcommand{\adj}{\mathrm{adj}}
\newcommand{\T}{\mathrm{T}}
\newcommand{\E}[2][]{\mathbb E_{#1} \!\left[ #2 \right]} % optional extra parameter #1
\newcommand{\bEu}[1]{\mathbf{e}_{#1}}
\newcommand{\bin}[2]{\text{Bin}(#1,#2)}
\newcommand{\Count}{1}
\newcommand{\constr}{\mathbf{G}}
\newcommand{\ind}{\mathbbm 1}
\newcommand{\ones}[1]{\mathbf{1}_{#1}}
\newcommand{\veth}{\boldsymbol{\hat \theta}}
\newcommand{\vbth}{\boldsymbol{\bar \theta}}
\newcommand{\iset}[1]{\lbrack #1 \rbrack}
\newcommand{\bxi}[1]{\mathbf{x}^{(#1)}}
\newcommand{\bXi}[1]{\mathbf{X}^{(#1)}}
\newcommand{\iomega}[1]{\omega^{(#1)}}
\newcommand{\bdi}[1]{\mathbf{d}^{(#1)}}
\newcommand{\argmax}{\operatornamewithlimits{argmax}}
% ----------------------------------------------------------------
\renewcommand{\labelenumi}{(\roman{enumi})}
\newcounter{tempcnt}
\renewcommand{\arraystretch}{1.2}
%-----------------------------------------------------------------

\def\gap{\vspace{10pt}\noindent}
\def\naive{na\"{\i}ve\ }



\def\bgamma{\boldsymbol{\gamma}}
% set two lengths for the includegraphics commands used to import the plots:
\newlength{\fwtwo} \setlength{\fwtwo}{0.45\textwidth}
% end of personal macros

\begin{document}
\DeclareGraphicsExtensions{.jpg}

\begin{center}
\textbf{\Large Fisher Information for Error Estimation Codes} \\[6pt]
  Paul Tune \\[6pt]
  School of Mathematical Sciences,\\
  University of Adelaide, Australia  \\[6pt]
  paul.tune@adelaide.edu.au
\end{center}

\begin{abstract}
In this technical report, we analyze the Fisher information of the generalized Error Estimating Codes (EEC) when the packet is sampled with 
and without replacement, in the case when the algorithm has immunity. We also compute the best bound possible obtained from looking 
directly at the Fisher information of transmitted bits.
\end{abstract}

\subparagraph{Key words.} Error estimation codes, Fisher information, sampling with and without replacement,
sketching.


\section{Introduction}


\section{Generalized EEC (gEEC)}

\subsection{Sampling with vs. without replacement}

Intuitively, it may seem that if one samples bits from the packet \textit{without replacement}, there is additional information in the subsketches 
due to the non-avoidance property of the sampling method. However, our aim in this section is to show that there is absolutely no difference
in the information content if \textit{sampling with replacement} was used instead.


\subsection{Computation of Fisher information}

Here we show a fast way of computing the Fisher information for gEEC, based on \cite[Lemma 1]{Hua12gEEC}. Let $\ast_K$ denote the circular 
convolution operation truncated at $K$
and $\bgamma(\theta,0) = [1,0,\cdots,0]$. Then, 
\be
\bgamma(\theta,\ell) = \bgamma(\theta,\ell-1) \ast_K m(\theta),
\ee
where $m(\theta) := [1-\theta,\theta/2,\cdots 0 \cdots,\theta/2]$. Clearly, $\bgamma'(\theta,0) = [0,0,\cdots,0]$. Similarly, by the chain rule, since 
the circular convolution is a linear operation (multiplication by a circulant matrix),
\be
\bgamma'(\theta,\ell) := \bgamma'(\theta,\ell-1) \ast_K m(\theta) + \bgamma(\theta,\ell-1) \ast_K m'(\theta),
\ee
where $m'(\theta) :=  [1,1/2,\cdots 0 \cdots,1/2]$. 

The Fisher information is simply
\be
J(\theta,\ell) = \sum_{k=0}^{K-1} J_k(\theta,\ell)
\ee
with
\be
J(\theta,\ell) := 
\begin{cases}
\frac{(\gamma'_k(\theta,\ell))^2}{\gamma_k(\theta,\ell)}, & \gamma_k(\theta,\ell) > 0,\\
0, & \text{otherwise}.
\end{cases}
\ee
We can compute these relations via publicly available circular convolution routines. Computational complexity of both $\bgamma(\theta,\ell)$ 
and $\bgamma'(\theta,\ell)$ are $O(K \log K)$ respectively.

 
\appendix

\bibliography{TuneBib}
\bibliographystyle{abbrv}

\end{document}

